

\documentclass{GfKl2006}

\usepackage{amsmath}
\usepackage{amsfonts}
\usepackage{latexsym}
\usepackage{times}

\begin{document}

\title*{Unbiased Recursive Partitioning II:\\
       A Parametric Framework Based on Parameter Instability Tests}

\author{
Achim Zeileis\inst{1}
\and
Torsten Hothorn\inst{2}
\and
Kurt Hornik\inst{1}
}


\institute{
Department f\"ur Statistik und Mathematik,
             Wirtschaftsuniversit\"at Wien \\
       Augasse 2-6, A-1090 Wien, Austria \\
       \texttt{Kurt.Hornik@wu-wien.ac.at} \\
       \texttt{Achim.Zeileis@wu-wien.ac.at}
\and
Institut f\"ur Medizininformatik, Biometrie und Epidemiologie\\
     Friedrich-Alexander-Universit\"at Erlangen-N\"urnberg\\
     Waldstra{\ss}e 6, D-91054 Erlangen, Germany \\
     \texttt{Torsten.Hothorn@rzmail.uni-erlangen.de}
}

\maketitle

\begin{abstract}
Recursive partitioning is a popular tool for
regression analysis. Two fundamental problems of exhaustive
search procedures usually applied to fit such models
have been known for a long time: overfitting and a selection bias towards
covariates with many possible splits or missing values. While pruning
procedures are able to solve the overfitting problem, the variable
selection
bias still seriously effects the interpretability of tree-structured
regression models. It is known for some time now that the variable
selection
problem can be solved by a separation of variable selection and cutpoint
estimation. For some special cases such unbiased procedures have been
suggested,
however lacking a common theoretical foundation.

In this pair of presentations, we propose two unified frameworks for
recursive partitioning with unbiased variable selection. In the first part,
a non-parametric procedure based on conditional inference techniques is suggested.
The second part is concerned with model-based partitioning utilizing 
parameter instability tests for variable selection.

Here, we extend the ideas from the first part to the case of recursively
partitioned parametric models. While the basic algorithm (1. association test
for each variable, 2. select variable with highest association, 3. select cutpoint)
remains the same, in each node a parametric model is fitted to the corresponding
subset. The association in Step~1 is assessed with generalized fluctuation tests
for parameter instability over each partitioning variable. This approach is applicable
to a wide class of parametric models, including in particular models estimated by
ordinary least squares and maximum likelihood, such as linear regression, generalized
linear models and survival regression. In addition, several strategies for 
cutpoint selection are discussed as well as pruning techniques based on information 
criteria.
\end{abstract}

\noindent
\textbf{Key words:} change points, maximum likelihood, parameter instability, recursive
partitioning

\end{document}
